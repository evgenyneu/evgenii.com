\documentclass[a4paper,twoside,12pt,hidelinks]{article}

% Packages
% ---------------------

\usepackage{amsmath} % Needed for command eqref
\usepackage{amsthm} % Used to show the proof box
\usepackage{fancyhdr} % Head and foot options
\usepackage{hyperref} % Uses automatic references \autoref
\usepackage[T1]{fontenc}
\usepackage{libertine}
\usepackage[libertine]{newtxmath} % Font

% Page settings
% ---------------------

\usepackage[top=2cm, bottom=2.5cm,left=2.5cm,right=2.5cm]{geometry} %  Page margins
\setlength{\parskip}{\baselineskip} % Add space between paragraphs
\setlength{\intextsep}{20pt plus 2.0pt minus 2.0pt} % Add vertical space before and after tables and figures (http://tex.stackexchange.com/a/26522/101976)
\parindent=0cm % Remove the paragraph indent of the first line
\addtolength{\jot}{2\jot} % Double the line between equations
\raggedbottom % Prevents spreading the page content vetically for non-full pages.

% Set vertical space around equations.
\AtBeginDocument{%
 \abovedisplayskip=15pt plus 5pt minus 5pt
 \abovedisplayshortskip=12pt plus 3pt
 \belowdisplayskip=15pt plus 5pt minus 5pt
 \belowdisplayshortskip=12pt plus 3pt minus 4pt
}


\begin{document}

\thispagestyle{empty}

\section*{Derivation of equation of motion for the angle $\theta$}

The derivation is used in the article \href{https://evgenii.com/blog/earth-orbit-simulation/}{https://evgenii.com/blog/earth-orbit-simulation}.

We have the Lagrangian
\begin{equation*}
  L = \frac{m}{2} (\dot{r}^2 + r^2 \dot{\theta}^2) + \frac{G M m}{r}.
\end{equation*}
We need to calculate
\begin{equation}
  \frac{d}{dt} \Big( \frac{\partial L}{\partial \dot{\theta}} \Big) = \frac{\partial L}{\partial \theta}.
  \label{eq_main_deribative}
\end{equation}
Since $L$ does not depend on $\theta$,
\[
  \frac{\partial L}{\partial \theta} = 0.
\]
Furthermore,
\begin{align*}
   \frac{\partial L}{\partial \dot{\theta}} &= \frac{\partial }{\partial \dot{\theta}} \Big( \frac{m}{2} r^2 \dot{\theta}^2 \Big) \\
    &= \frac{m}{2} r^2 (2) \dot{\theta} = m r^2 \dot{\theta}.
\end{align*}
Next, we take the time derivative:
\begin{align*}
   \frac{d}{dt} \Big( \frac{\partial L}{\partial \dot{\theta}} \Big) &= \frac{d}{dt} \Big( m r^2 \dot{\theta} \Big) \\
    &= m \frac{d}{dt} \Big( r^2 \dot{\theta} \Big) \\
    &= m (2 r \dot{r} \dot{\theta} + r^2 \ddot{\theta}). \tag{Product rule}
\end{align*}
Substituting into \autoref{eq_main_deribative} gives
\[
  m ( 2 r \dot{r} \dot{\theta} + r^2 \ddot{\theta} ) = 0.
\]
Simplify:
\[
  2 \dot{r} \dot{\theta} + r \ddot{\theta} = 0.
\]
Finally, we solve for $\ddot{\theta}$:
\[
  \ddot{\theta} = - \frac{2 \dot{r} \dot{\theta}}{r}.
\]


\end{document}